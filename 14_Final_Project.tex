\documentclass{beamer}
\usepackage{graphicx} % Required for inserting images

\usetheme{PaloAlto}
\usecolortheme{seahorse}

\title{The fire that hit Tenerife in august 2023}
\author{Anis Edena Zarantonello}
\date{10 July 2024}

\begin{document}

\maketitle % is a prefixed way to make the title, is necessary if we want uniform titles 

\AtBeginSection[] 
{
\begin{frame}
\frametitle{Outline}
\tableofcontents[currentsection] % to do an index at the beginning of every section to make evident the new section.
\end{frame}
}


\section{Introduction}
\begin{frame}{Introduction}
\begin{itemize}
    \item August 15 - September 5
    \item A "sixth generation wildfire"
    \item Intentioned fire, but Global Warming as the main driver
    \item 15.000 hectares, 12 municipalities, great natural damages
\end{itemize}
\begin{columns}
\begin{column}{0.5\textwidth}
\begin{center}
    \includegraphics[width=\textwidth]{ten1.jpg}
\end{center}
\end{column}
\begin{column}{0.5\textwidth}  
    \begin{center}
     \includegraphics[width=\textwidth]{ten2.jpg}
     \end{center}
\end{column}
\end{columns}
\end{frame} 

\section{Study Area}
\begin{frame}{Study Area}
Tenerife - Canary Islands
\begin{columns}
\begin{column}{0.5\textwidth}
\begin{center}
    \includegraphics[width=\textwidth]{canarias.png}
\end{center}
\end{column}
\begin{column}{0.5\textwidth}  
    \begin{center}
     \includegraphics[width=\textwidth]{Tenerife.png}
     \end{center}
\end{column}
\end{columns}
    
\end{frame}


\section{Aim of the project}

\begin{frame}{Aim of the project}
\begin{itemize}
    \item Analysis of the area damaged at the end of the fire
    \item Comparison: July 9 - September 7
\end{itemize}
\begin{column}{1\textwidth}
    \begin{center}
        \includegraphics[width=\textwidth]{Tenerife_settembre.jpg}
    \end{center}
\end{column}
\end{frame}

\begin{frame}{Aim of the project}
Analysis of the following indices:
\begin{itemize}
    \item \begin{equation}
NBR=\frac{NIR-SWIR}{NIR+SWIR}
    \end{equation}
    \item  \begin{equation}
dNBR=prefireNBR - postfireNBR
    \end{equation}
\end{itemize}
between days july 9 and september 7, to compare the pre fire and post fire situation
\end{frame}

\begin{frame}{What is the Normalized Burn Ratio (NBR)?}
\includegraphics[width=\textwidth]{spectrum NBR.png}
\end{frame}

\begin{frame}{Bands reflectance}
\includegraphics[width=\textwidth]{Julybands.jpeg}
\includegraphics[width=\textwidth]{Septemberbands.jpeg}
\end{frame}


\section{Matherials and methods}

\begin{frame}{Materials and methods}
\begin{itemize}
    \item Copernicus browser: Sentinel-2, bands NIR (b8), SWIR(b12)
    \item R Studio for data analysis
\end{itemize}
% dato che nn viene farei due foto e via
\begin{columns}
    \begin{column}{0.5\textwidth}
        \includegraphics[width=\textwidth]{packages.png}
    \end{column}
    \begin{column}{0.5\textwidth}
    \includegraphics[width=\textwidth]{functions.png}
    \end{column}
\end{columns}
\end{frame}

\section{Results: overview}
\begin{frame}{NBR overview}
\includegraphics[width=\textwidth]{NBRdifference.jpeg}
\end{frame}

\begin{frame}{dNBR overview}
  \includegraphics[width=\textwidth]{dNBR.jpeg}  
\end{frame}

\section{Results in the study area}
\begin{frame}{Results in the study area: NBR}
\includegraphics[width=\textwidth]{NBRzona2.jpeg}
\end{frame}

\begin{frame}{Results in the study area: dNBR}
\includegraphics[width=\textwidth]{dNBRzona.jpeg}
\end{frame}

\begin{frame}{Results in the study area}
\begin{columns}
    \begin{column}{0.5\textwidth}
        \includegraphics[width=\textwidth]{Julyclassified.jpeg}
        \includegraphics[width=\textwidth]{pixelsj.png}
    \end{column}
    \begin{column}{0.5\textwidth}
    \includegraphics[width=\textwidth]{Septclassified.jpeg}
    \includegraphics[width=\textwidth]{pixelss.png}
    \end{column}
\end{columns}
\end{frame}


\begin{frame}{Land cover changes}
\includegraphics[width=\textwidth]{classification.jpeg}
\end{frame}


\section{Discussion and conclusions}
\begin{frame}{Discussion and conclusions}
\begin{itemize}
    \item The results obtained with this analysis highlight a loss of vegetation cover caused by the fire.
    \item NBR index values decrease from +1 to -1 comparing june to september, while dNBR shows values near +1, indicating a high burn severity.  
    \item Fires are happening more and more and this leads to huge environmental impacts having significant natural and social consequences.
\end{itemize}
    
\end{frame}

    

\end{document}
