\documentclass[12pt]{article} % Indicates the type of document we are doing, in this case an aticle with character dimension of 12.

% packages always at the beginning
\usepackage{graphicx} % to load images 
\usepackage{hyperref} % to create linkages inside the text 
\usepackage{natbib} % to organize bibliography
\usepackage{lineno} % to number rows of the text
\usepackage{color}

% \ are the functions that have {}, inside of which we can put the arguments, [] are used for stilish or structural modifications.

\linenumbers % function of lineno package that number the rows of the text we are writing.

% Starting document: titol, author and date 

\begin{document} % after a begin there is always an end that end the operation.

\title{My first LaTeX doc - Jigsaw Falling into Place}
\author{Anis Zarantonello}
\date{11 Maggio 2024} % permit to change the settings (american or european) of the date depending on the aruments that have been passed. If not present, it takes directly the date from the system setted as mm/gg/aa. 

\maketitle % takes pieces before begin and write them in the document (it's necessary also if they are after the begin: I think is for making the title)
\tableofcontents % it creates an authomatic index that is upload everytime I add something.

% Abstract

\begin{abstract} %  
This song offer a glimpse into the complexities of human interaction and the desire for connection.
\end{abstract}

\bigskip % I use this function to create a bigger space; we also could use \\ at the end of the precious row. 
% I could use \smallskip if I want a smaller space

\textit{Keywords}: Introspection, Emotions % Function that adds the key word section in my article. Using \textit words will be in cursive

\section{\textit{Introduction}} % Function \section creates a new section in the document, in this way I am starting the introduction. I can tell Latex to write in cursive using \textit. To remove the number before the section we have to put * before curly brackets. (in this way it disappears from the index and we don't know why?!)

\label{sec:intro} % I assign a label to the section (in this case I call it "intro") to call it beck in the text with \ref{} 

% If you make a mistake in the name of the section, ?? will appear in the portion in which I call back \ref{}. Thus to understand if they are errors in the document in a simple way. 

\textbf{Jigsaw} % putting in bold type the portion in the text that I write in the argument 
Jigsaw Falling into Place by Radiohead is a song that encapsulates the exhilaration, introspection, and mixed emotions of a night out. \cite{Paul2010} % function needed to cite something from the bibliography; I have to create the bibliography and give it a lable that than I can use to call it back in the text. So before I have to create the bibliography section.
\citep{Paul2010} % as before ?
\smallskip

% Come si fa ad andare a capoo?

\citet{Paul2013} % in this case I cite the bibliography as a text only with the year inside the brackets.
The lyrics of Jigsaw Falling into Place paint a vivid picture of a night out, capturing the excitement, chaos, and emotional rollercoaster that often accompanies such moments.
\footnote{In reality Paul have not said neither of these things} % to insert some notes at the end of the page
\bigskip

To view a YouTube video that has absoloutly nothing to do with our study: 
\url{https://www.youtube.com/watch?v=Hobdg-Fp9kI} % inserting a link

\newpage % adding a new page

\section{Methods} % creating Methods section
This study is the best result of my spontaneous imagination during a listening session in the gardens

\subsection{Study Area} % creating a subsection
Personally I really love this song, I think it's crazy

\subsection{Algoritms}

% Writing an equation
% We can type in the web LaTeX Math and try to find all the various sections.

For this study we used some equations (mathematical of course)\footnote{Of course we are joking.. no equation needed in serious studies}

\ref{eq:sum}: % I call back the equation that I create below

\begin{equation} % creating the equation
    T = \sum p_i % \sum insert the sum symbol, then I put what I want
\label{eq:sum} % I can attribute a label to the eq. so that I can call it back 
\end{equation} % End of it 

% another equation
Obviously there is another important equation due to the fact that Newton loved Radiohead (probably yes if he had known them) \ref{eq:newton}:
\begin{equation}
    F = \sqrt[2]{G \frac{m_1\times m_2}{d^2}} 
    % \sqrt to put under root, [] to put the power (exponent) of the root 
    % \frac to do the fraction (curly brackets with numerator and curly brackets with denominator)
    % _ necessary to write the subscript of the arguments
    % \times{} multiplication function. 
\label{eq:newton}
\end{equation}

\newpage

\section{Results}
Just as you take my hand. Just as you write my number down. Just as the drinks arrive. Just as they play your favourite song. As your bad day disappears. No longer wound up like a spring. Before you've had too much. Come back in focus again. The walls are bending shape. They've got a Cheshire cat grin. All blurring into one. This place is on a mission. Before the night owl. Before the animal noises. Closed circuit cameras. Before you're comatose.

\section{Discussion}
No explaination needed. \ref{sec:intro} % I call back the lable of the intro section 

We where able to observe: 
\begin{itemize} % Function to create a bulletted list (the argument is itemize)
    \item Hand 
    \item Number 
    \item Drink
    \item Favourite song 
\end{itemize} % end of the list

Best 2 observations: \footnote{Astronauts making love was not so funny to look at..}
\begin{enumerate} % The same function but with enumerate as argument do an enumerated list
    \item Hand
    \item Favourite song 
\end{enumerate}

\newpage 

One of the best images we where able to take from the basement is \ref{fig:Rad}

% adding an image 
\begin{figure}
    \centering % centering the image in the page
    \includegraphics[width=\textwidth]{space.jpg} % [] in this way the image is wide as the text is. If we want half wide we would write width=0.5. The image has to be downloaded in the PC and call it back with the name of the extention.
    \caption{The band in the basement} % inserting a caption for the image
    \label{fig:Rad} % to call it back in the text
\end{figure}


\bigskip
\bigskip
\bigskip
\bigskip
\bigskip
\bigskip
\bigskip
\bigskip
\bigskip
% let's do a table
\hline % drawing an horizontal line (horizontal)
\smallskip
\textbf{IMPORTANT}

\bigskip
\textbf{NO musicians were hurt during this study.}
\textit{at least not yet..}
\smallskip
\hline % I close the box drawing another line
% maybe some errors that we don't get

\newpage

\begin{thebibliography}{999}
    \bibitem[Paul,2010]{Paul2010} % inside [] i put what I want that appears in the text while inside {} i put in which way I want to call, so the lable to call it back.
    Paul, E. (2010). Let's all love Radiohead.
    
    \bibitem[Paul et al., 2013]{Paul2013}
    Paul, A. L., Wheeler, R. M., Levine, H. G., \& Ferl, R. J. (2013). I have no more ideas to convince you listening to them.
\end{thebibliography}

\end{document}
